\documentclass[10pt, conference, compsocconf]{styles/IEEEtran}
\usepackage{cite}
\usepackage{balance}

\usepackage[pdftex]{graphicx}
\usepackage{grffile} % multiple dots in filename
\graphicspath{{figures/}}
\DeclareGraphicsExtensions{.pdf}

\usepackage[cmex10]{amsmath}
\interdisplaylinepenalty=2500
% \usepackage{algorithmic}
%\usepackage[caption=false,font=footnotesize]{subfig}
\usepackage{subfig}
\usepackage{wrapfig}

\usepackage{url}
\usepackage[usenames,dvipsnames]{color}

% should be loaded last (but before algorithm*) -colored links and refs
\usepackage[colorlinks=true, citecolor=OliveGreen, linkcolor=BrickRed, urlcolor=MidnightBlue, final, pdftex]{hyperref}
%\hypersetup
%{
%    pdfauthor={Authors},
%    pdfsubject={Subject},
%    pdftitle={Title},
%    pdfkeywords={Keywords}
%}

\usepackage{xspace}
\usepackage{algorithm}
\usepackage{algorithmic}

\setlength{\parskip}{0ex}

%% Using watermark (this will mess up top/bottom margins)
%\usepackage{styles/pdfdraftcopy}
%\draftcolor{gray20}
%\draftstring{
%\begin{minipage}{17cm}
%\begin{center}
%\  DRAFT - \today\\Not approved for public release
%\end{center}
%\end{minipage}
%}
%\draftfontsize{36pt}

% Another option; doesn't clobber margins, but doesn't seem to like multiple lines

%\usepackage{draftwatermark}
%\SetWatermarkText{DRAFT of \today. Not approved for public release}
%\SetWatermarkScale{.2}
%\SetWatermarkColor[rgb]{0.7,0.7,0.7}

% custom commands
\newcommand{\etal}{{\em et al.}}
\newcommand{\naive}{na\"{\i}ve }
%\newcommand{\point}[1]{\vspace{2mm} \noindent\textbf{#1}}
\newcommand{\point}[1]{\noindent\textbf{#1}.}
\newcommand{\todo}[1]{\textbf{\color{red}[TODO: #1]}}
\newcommand{\rob}[1]{\textbf{\color{blue}[rob: #1]}}
\newcommand{\citeme}{\textbf{\color{red} {cite}}\xspace}
\renewcommand{\S}{Section~}
\newcommand{\Eg}{\emph{E.g.}}
\newcommand{\eg}{\emph{e.g.}}
\newcommand{\etc}{\emph{etc}}
\newcommand{\Ie}{\emph{I.e.}}
\newcommand{\ie}{\emph{i.e.}}
\newcommand{\paul}[1]{{\color{red}\em (Paul says: ``#1'')}}

\newcommand{\ps}{TAPS\xspace}
\newcommand{\compactify}{\settowidth{\labelsep}{o} \settowidth{\labelwidth}{o} \settowidth{\labelindent}{o}}

\title{Genuine onion: Simple, Fast, Flexible, and Cheap Website Authentication and Security}
%\title{Simple, Fast, Flexible, and Cheap Website Authentication,
%  Integrity, and Traffic Security}

\author{
\IEEEauthorblockN{Paul Syverson}
\IEEEauthorblockA{U.S. Naval Research Laboratory\\
paul.syverson@nrl.navy.mil}
\and
\IEEEauthorblockN{Griffin Boyce}
\IEEEauthorblockA{Open Internet Tools Project\\
griffin@cryptolab.net}
}

\begin{document}

\maketitle

\begin{abstract}
  Use of Tor's .onion virtual domain has traditionally focused on
  offering hidden services, services that separate their reachability
  from the identification of their IP addresses. We argue that Tor's
  Hidden Service Protocol and .onion addresses can be used to provide
  an entirely separate benefit: basic website authentication. We also
  argue that not only can onionsites provide website authentication,
  but doing so is easy, fast, cheap, flexible and secure when compared
  to alternatives such as TLS with certificates.
\end{abstract}

% \input{sections/abstract}
% \begin{IEEEkeywords}
% anonymity; throttling; experimentation; Tor;
% \end{IEEEkeywords}
% \input{sections/introduction}
% \input{sections/Brief background on Tor and .onion services}
% \input{sections/Knowing to which self to be true}
% \input{sections/Our onions ourselves}
% \input{sections/Considerations for Hidden Service Operators}
% \input{sections/performance}
% \input{sections/errors}
% \input{sections/propagation}
% \input{sections/related}
% \input{sections/conclusion}

\section{Introduction}
Tor is a widely popular communications infrastructure for anonymous
communication. Millions use its thousands of relays for unfettered
traffic-secure access to the Internet. The vast majority of Tor
traffic by bandwidth (over 99\% at our last check) is on circuits
connecting Tor clients to servers that are otherwise accessible.  Tor
also provides protocols for connecting to services on the
pseudo-top-level domain .onion, which are only accessible via Tor.

In this paper we explore using Tor's .onion infrastructure so that
individuals operating a website can create authentication, integrity
and other guarantees more simply, easily, fully, cheaply and/or flexibly
than by relying on standard protocols and authorities.

Tor's onionsites have been advocated since their introduction as a
way to protect network location information for servers not just
clients~\cite{tor-design}. \footnote{Such advocacy actually predates
  their introduction inasmuch as the same was said for web servers
  contacted by reply onions~\cite{onion-routing:cacm99}, See
  also~\cite{rewebber} for description of an implemented predecessor
  to Tor's hidden services.}  Discussion in the popular press, as well
as research to date, has focused almost exclusively on location hiding
and associated properties provided by onionsites and the protocols
to interact with them. Indeed, these are generally referred to
collectively as \emph{Tor Hidden Services} in the research literature
and as the \emph{Dark Web} in the popular press. (Although so many
importantly distinct things are often subsumed and run together under
`Dark Web' as to rob the term of clear significance, other than as a
warning flag to watch out for the hidden incoherence that surrounds
most occasions of its use.)

Our intent is to challenge the narrowness of this view of
onionsites. In particular we will discuss security protections they
readily facilitate that are largely orthogonal to hiding server
location. We hope by the end of this paper, the reader will agree that
they should more properly be called \emph{.onion services} or in any
case something that is more properly inclusive of the variety of
security properties they offer.

\section{Brief background on Tor and .onion services}

We sketch out minimal basics of Tor .onion services. For more detailed
descriptions see the Tor design paper or other documentation at the
Tor website~\cite{torproject}. For a high-level graphical description
of .onion services see~\cite{tor-hs}. For a more up to date, and much
more technical, description of how .onion services work see the Tor
Rendezvous Specification~\cite{tor-rend-spec}.

Tor clients randomly select three of the roughly 6000 relays
comprising the current Tor network, and create a cryptographic circuit
through these to connect to Internet services. Since only the first
(entry guard) relay in the circuit sees the IP address of the client and
only the last (exit) relay sees the IP address of the destination,
this technique separates identification from routing.

To offer a .onion service, a web server creates Tor circuits to
multiple \emph{Introduction Points} that await connection attempts
from clients. A user wishing to connect to a particular .onion service
uses the .onion address to look up these Introduction Points in a
directory system. In a successful interaction, the client and .onion
site then both create Tor circuits to a client-selected
\emph{Rendezvous Point}. The Rendezvous Point mates their circuits
together, and they can then interact as ordinary client and server
of a web connection over this rendezvous circuit.

Since the onionsite only communicates over Tor circuits it creates,
this protocol hides its network location. And that is the feature that
gives it the name 'hidden service'. But, there are other important
features to the .onion system, notably self-authentication. The .onion
address is actually the hash of the public key of the onionsite. For
example, if one wishes to connect to the DuckDuckGo search engine's
.onion service, the address is 3g2upl4pq6kufc4m.onion. The Tor client
recognizes this as a .onion address and thus knows to use the above
protocol rather than attempting to pass the address through a Tor
circuit for DNS resolution at the exit. The public key
corresponds to the key that signs the list of of Introduction Points
and other service descriptor information provided by the directory
system. In this way, .onion addresses are self-authenticating.

\section{Knowing to which self to be true}

Of course this authentication only binds the service descriptor
information to the 3g2upl4pq6kufc4m.onion address. What a user would
like to be assured of is that s/he is reaching DuckDuckGo. Presumably
the user wants the search results DuckDucGo offers and not what might
be returned by some other, possibly malicious, server.  In addition to
the integrity guarantee, the user relies on authentication so that its
queries are revealed only to DuckDuckGo and not to others. The
.onion address by itself does not offer this.
Making use of the traditional web trust infrastructure, DuckDuckGo
offers a certificate for its .onion address issued by DigiCert.
Similarly, FaceBook offers a DigiCert certificate.

Though cryptographic binding is essential to the technical mechanisms
of trust, users also rely on human-readable familiarity. that their
browser graphically indicates they have made a certified encrypted
connection as a result of typing facebook.com into the browser.  To
some extent, it is at least possible to make use of this in .onion
space. By generating many keys whose hash had 'facebook' as initial
string and then looking among the full hashes for an adequately
felicitous result, Facebook was able to obtain facebookcorewwwi.onion
for its address. Whatever its value for Facebook, this is clearly not
something that will work widely.

Why not just obtain certificates from traditional authorities as
DuckDuckGo and Facebook have done? For many server operators, getting
even a basic server certificate is just too much of a hassle. The
application process can be confusing. It usually costs money. It's
tricky to install correctly. It's a pain to update. This is not an
original observation. Indeed it is actually a quote from the blog of
Let's Encrypt, a new certificate authority dedicated among other
things to making TLS certification free and
automatic~\cite{lets-encrypt}.

Should they be willing to offer certificates for .onion domains, using
Let's Encrypt would be an easy way for onionsite operators to take
advantage of the traditional certification infrastructure. Traditional
certificates are not without problems, however. The nature of the
trust hierarchy is opaque to ordinary users, and the sheer number of
trusted authorities is large enough to be of concern. In particular,
there have been numerous documented cases of governmental
man-in-the-middle attacks through certificate manipulation and hacking
of certificate authorities leading to use of fraudulent certificates
for some of the most popular websites. 

The EFF's SSL Observatory~\cite{ssl-observatory} is designed to
monitor for such problems and document their occurrence, while
Google's Certificate Transparency
Effort~\cite{certificate-transparency} is a similar but broader
initiative that adds (amongst other things) append-only signed public
logs that make certificate shenanigans all the harder to bring off
undetectably.

Rather than simply monitor and make available certificate authority
problems, the Perspectives Project~\cite{perspectives}, strives to
provide end user's with control over the trust they place in website
certificates through its Firefox Extension. Instead of trusting
anointed CAs, semi-trusted network notaries probe network services and
build a record of public keys those services have used over time
somewhat similar to the approach of certificate transparency. Users
can choose which notaries they wish to trust, and clients encountering
unfamiliar public keys will query notaries for a history of keys used
by a service~\cite{perspectives-paper}. This is especially intended to
enhance trust on first use (tofu) authentication, although it can also
supplement traditional CA based PKI security. While Perspectives
provides security over time for self-signed certificates, it is still
subject to tofu assumptions and dependent on the setup of an adequate
independent notary system.

\section{Our onions ourselves}

As noted onionsites already provide a self-authenticed binding of
public key to onion address, but lack a way to bind that public key to
something authoritative for the site in question.  Even for hidden
service applications, it might still be desired to connect the
onionsite to some sort of pseudonymous reputation.  For us, however,
the hiding aspect is largely orthogonal.  We would like a solution
that will work for all kinds of sites, but we are especially
interested in providing authentication for small and/or shortlived
websites, e.g., personal webpages, hometown sports teams, sites for
local one-time events, small businesses, municipal election campaigns,
etc.  Though not such large targets as more popular sites, they are
still subject to controversy and have been subject to many of the same
sorts of attacks as more wellknown sites.  Some users of this kind may
not even have Internet accounts that allow them to set up
servers. Onionsites are compliant with such a requirement since they
actually only make outbound client connections.
Even if the user has an Internet account that permits setting up
a webpage, HTTPS may not be available from that provider or only
available for an additional fee. 

Of course one can always create a Facebook page or something similar,
and this is often done.  But this makes the service dependent on the
reputation, trust, policies, and protections of such a host and the
dynamics thereof, rather than allowing the user to readily understand
and control these aspects of his own service. Also, Facebook
policy requires identification of the person providing the site
while we would prefer to leave this as simply a separate question.

Perhaps the simplest way to add binding of the onionsite public key to
a known entity using widely available mechanisms is to provide a
signature on the .onion address. We envision a PGP/GPG signature, but
it could be an X509 signature or other. The signed text can simply be
included on the onionsite, making it self-authenticating in this sense
as well. The trust in the authentication will then be whatever trust
is associated with the public key that does the signing. Such
techniques are already used for signing code. For example,
the Tor Project offers signatures on all source and binaries
it makes available for download.

If the signer wishes to post the signed .onion address to a Facebook
page or other public site, s/he can do this also. Indeed, a not
unlikely scenario would be one in which one has a public website
available via HTTP with a signed pointer to a secure and authenticated
version of the same service available via an onionsite. It is then
easy for anyone who desires to check if the unauthenticated version
matches the secure version. For example, we have made an authenticated
version of http://cupcakebridge.com available at http://?????.onion
\paul{Griffin, is this something you could do relatively easily so we
  can point at it by time of submission? I fugure it's also an easy
  way to put in a side plug for cupcake. If not, could you suggest an
  alternative?}
\griffin{Yes, I can set that up.}

Note that while Perspectives offers an improvement against certificate
based MitM attacks, the approach just described is more resistant to
MitM than Perspectives. First, it relies on available trust
information in the PKI (e.g., PGP web of trust) rather than a history
of behavoir recorded by notaries.  Thus, if a site is newly available
the onionsite can still be trusted to be bound to the signing
authority whereas a self-signed certificate will have little or no
Perspective history and users are reduced to effectively a tofu
decision. Second, the Perspective notaries largely function as
detectors of misbehavior. Thus if, e.g., a typosquatting site uses a
self-signed certificate to pass through connections to the site on
which it is squatting, but does not misbehave or alter its key,
Perspectives will not reflect anything wrong. If the attacked site
has a signed certificate that warrants a graphical indicator in
the browser, an ordinary user may notice it's absence. But Perspectives
does not add to that protection.
***Add more here and clean up.***


Our approach also can be used right now as opposed to
requiring new infrastructure and can rely on whatever authentication
infrastructure might be popular and likely to be maintained for
independent reasons, rather than needing to grow and maintain
interest in its approach.


Still to do: GSA use etc.



%\point{Acknowledgments}
% \section*{Acknowledgments}
% We thank the anonymous reviewers for their feedback and suggestions.

% trigger a \newpage just before the given reference
% number - used to balance the columns on the last page
% adjust value as needed - may need to be readjusted if
% the document is modified later
%\IEEEtriggeratref{22}
% The "triggered" command can be changed if desired:
%\IEEEtriggercmd{\enlargethispage{-5in}}
 
\newcommand{\BIBdecl}{\setlength{\itemsep}{0\baselineskip plus 0.1\baselineskip minus 0.1\baselineskip}}
\balance
{\footnotesize 
\bibliographystyle{styles/IEEEtran}
\bibliography{references}
}

%\nocite{*}
 
%\clearpage
%\input{sections/appendix}

\end{document}


Furthermore, these are often the sorts of users least able to configure
and secure 





There are a number of problems and limitations with the existing
web authentication and certification infrastructure, most of which
we will not touch on. There are, however, several that can be
improved upon through use of onionsites.

Unlike TLS, key management is extremely straightforward for .onion
service operators, and key exchange is handled by the network itself.
The key size and cipher have rational defaults.  The downside is that
anyone with the key can impersonate an onion site, as the .onion
address is inextricably tied to its key.  An adversary who acquires a
TLS private key cannot impersonate a clearnet website, as the key is
unrelated to the domain name.


\section{Other Considerations for Hidden Service Operators}



