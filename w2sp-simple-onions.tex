\documentclass[10pt, conference, compsocconf]{styles/IEEEtran}
\usepackage{cite}
\usepackage{balance}

\usepackage[pdftex]{graphicx}
\usepackage{grffile} % multiple dots in filename
\graphicspath{{figures/}}
\DeclareGraphicsExtensions{.pdf}

\usepackage[cmex10]{amsmath}
\interdisplaylinepenalty=2500
% \usepackage{algorithmic}
%\usepackage[caption=false,font=footnotesize]{subfig}
\usepackage{subfig}
\usepackage{wrapfig}

\usepackage{url}
\usepackage[usenames,dvipsnames]{color}

% should be loaded last (but before algorithm*) -colored links and refs
\usepackage[colorlinks=true, citecolor=OliveGreen, linkcolor=BrickRed, urlcolor=MidnightBlue, final, pdftex]{hyperref}
%\hypersetup
%{
%    pdfauthor={Authors},
%    pdfsubject={Subject},
%    pdftitle={Title},
%    pdfkeywords={Keywords}
%}

\usepackage{xspace}
\usepackage{algorithm}
\usepackage{algorithmic}

\setlength{\parskip}{0ex}

%% Using watermark (this will mess up top/bottom margins)
%\usepackage{styles/pdfdraftcopy}
%\draftcolor{gray20}
%\draftstring{
%\begin{minipage}{17cm}
%\begin{center}
%\  DRAFT - \today\\Not approved for public release
%\end{center}
%\end{minipage}
%}
%\draftfontsize{36pt}

% Another option; doesn't clobber margins, but doesn't seem to like multiple lines

%\usepackage{draftwatermark}
%\SetWatermarkText{DRAFT of \today. Not approved for public release}
%\SetWatermarkScale{.2}
%\SetWatermarkColor[rgb]{0.7,0.7,0.7}

% custom commands
\newcommand{\etal}{{\em et al.}}
\newcommand{\naive}{na\"{\i}ve }
%\newcommand{\point}[1]{\vspace{2mm} \noindent\textbf{#1}}
\newcommand{\point}[1]{\noindent\textbf{#1}.}
\newcommand{\todo}[1]{\textbf{\color{red}[TODO: #1]}}
\newcommand{\rob}[1]{\textbf{\color{blue}[rob: #1]}}
\newcommand{\citeme}{\textbf{\color{red} {cite}}\xspace}
\renewcommand{\S}{Section~}
\newcommand{\Eg}{\emph{E.g.}}
\newcommand{\eg}{\emph{e.g.}}
\newcommand{\etc}{\emph{etc}}
\newcommand{\Ie}{\emph{I.e.}}
\newcommand{\ie}{\emph{i.e.}}

\newcommand{\ps}{TAPS\xspace}
\newcommand{\compactify}{\settowidth{\labelsep}{o} \settowidth{\labelwidth}{o} \settowidth{\labelindent}{o}}

\title{Simple, Fast, Flexible, and Cheap Website Authentication,
  Integrity, and Traffic Security}

\author{
\IEEEauthorblockN{Paul Syverson}
\IEEEauthorblockA{U.S. Naval Research Laboratory\\
paul.syverson@nrl.navy.mil}
\and
\IEEEauthorblockN{Griffin Boyce}
\IEEEauthorblockA{Institution1\\
Email2}
}

\begin{document}

\maketitle

\begin{abstract}
  Use of Tor's .onion virtual domain has traditionally focused on offering
  \emph{hidden services}, services that separate their reachability
  from the identification of their IP addresses. This position paper
  argues that Tor's Hidden Service Protocol and .onion addresses
  provide an entirely separate benefit: website authentication that is
  easy, fast, cheap and flexible to configure, deploy, and use.
\end{abstract}

% \input{sections/abstract}
% \begin{IEEEkeywords}
% anonymity; throttling; experimentation; Tor;
% \end{IEEEkeywords}
% \input{sections/introduction}
% \input{sections/model}
% \input{sections/metrics}
% \input{sections/algorithm}
% \input{sections/experiments}
% \input{sections/performance}
% \input{sections/errors}
% \input{sections/propagation}
% \input{sections/related}
% \input{sections/conclusion}

\section{Introduction}
In this paper we explore using Tor's .onion infrastructure so that
individuals operating a website can create authentication, integrity
and other guarantees more simply, easily, fully, cheaply and/or flexibly
than by relying on standard protocols and authorities.

Tor's .onion sites have been advocated since their introduction as a
way to protect network location information for servers not just
clients~\cite{tor-design}. \footnote{Such advocacy actually predates
  their introduction inasmuch as the same was said for web servers
  contacted by reply onions~\cite{onion-routing:cacm99}, See
  also~\cite{rewebber} for description of an implemented predecessor
  to Tor's hidden services.}  Discussion in the popular press, as well
as research to date, has focused almost exclusively on location hiding
and associated properties provided by .onion sites and the protocols
to interact with them. Indeed, these are generally referred to
collectively as \emph{Tor Hidden Services} in the research literature
and as the \emph{Dark Web} in the popular press. (Although so many
importantly distinct things are often subsumed and run together under
`Dark Web' as to rob the term of clear significance, other than as a
warning flag that there may be a high probability of incoherent nonsense
in what follows.)

Our intent is to note the narrowness of this view of .onion sites. In
particular we will discuss security protections they readily
facilitate that are largely orthogonal to hiding server location. We
hope by the end of this paper, the reader will agree that they should
more properly be called \emph{.onion services} or in any case
something that is more properly inclusive of the variety of security
properties they offer.

\section{Brief background on Tor and .onion services}

We sketch out the basics of Tor and .onion services. For more detailed
descriptions see the Tor design paper or other documentation at the
Tor website~\cite{torproject}. For a high-level graphical description
of .onion services see~\cite{tor-hs}. For a more up to date, and much
more technical, description see the Tor Rendezvous
Specification~\cite{tor-rend-spec}.





%\point{Acknowledgments}
% \section*{Acknowledgments}
% We thank the anonymous reviewers for their feedback and suggestions.

% trigger a \newpage just before the given reference
% number - used to balance the columns on the last page
% adjust value as needed - may need to be readjusted if
% the document is modified later
%\IEEEtriggeratref{22}
% The "triggered" command can be changed if desired:
%\IEEEtriggercmd{\enlargethispage{-5in}}
 
\newcommand{\BIBdecl}{\setlength{\itemsep}{0\baselineskip plus 0.1\baselineskip minus 0.1\baselineskip}}
\balance
{\footnotesize 
\bibliographystyle{styles/IEEEtran}
\bibliography{references}
}

%\nocite{*}
 
%\clearpage
%\input{sections/appendix}

\end{document}
